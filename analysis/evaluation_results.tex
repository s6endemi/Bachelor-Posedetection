\documentclass[11pt,a4paper]{article}
\usepackage[utf8]{inputenc}
\usepackage[margin=2.5cm]{geometry}
\usepackage{booktabs}
\usepackage{graphicx}
\usepackage{xcolor}
\usepackage{hyperref}
\usepackage{float}
\usepackage{caption}
\usepackage{subcaption}
\usepackage{amsmath}
\usepackage{siunitx}
\usepackage{multirow}
\usepackage{colortbl}
\usepackage{enumitem}
\usepackage{titlesec}

% Colors
\definecolor{bestcolor}{RGB}{200, 230, 200}
\definecolor{worstcolor}{RGB}{255, 220, 220}
\definecolor{neutralcolor}{RGB}{240, 240, 240}

\titleformat{\section}{\large\bfseries}{\thesection}{1em}{}
\titleformat{\subsection}{\normalsize\bfseries}{\thesubsection}{1em}{}
\setlength{\parindent}{0pt}
\setlength{\parskip}{0.5em}

\title{\textbf{Pose Estimation Evaluation}\\[0.3em]\large Vollständige Ergebnisübersicht}
\author{Eren}
\date{Januar 2026}

\begin{document}
\maketitle

%==============================================================================
\section*{Executive Summary}
%==============================================================================

\begin{table}[H]
\centering
\begin{tabular}{@{}ll@{}}
\toprule
\textbf{Erkenntnis} & \textbf{Details} \\
\midrule
Beste Genauigkeit & MoveNet (11.5\% NMPJPE bereinigt) \\
Robusteste Selection & MediaPipe ($\sim$2x besser bei Multi-Person) \\
Rotationseffekt & +27--38\% Fehleranstieg bei seitlicher Ansicht \\
Hauptproblem & c17-Kamera hat 10x mehr Person-Switch-Frames \\
\bottomrule
\end{tabular}
\end{table}

\textbf{Datenbasis:} 363.529 Frames aus 126 Videos (REHAB24-6 Dataset)

%==============================================================================
\section{Datenbasis und Methodik}
%==============================================================================

\subsection{Dataset}

\begin{table}[H]
\centering
\begin{tabular}{@{}ll@{}}
\toprule
Eigenschaft & Wert \\
\midrule
Dataset & REHAB24-6 (Zenodo) \\
Videos & 126 (21 Patienten $\times$ 6 Übungen) \\
Kameras & c17 (frontal), c18 (lateral) \\
Analysierte Frames & 363.529 \\
Frame-Step & 3 (jeder 3. Frame bei 30fps = 10Hz) \\
Ground Truth & Motion Capture (optische Marker) \\
\bottomrule
\end{tabular}
\caption{Dataset-Übersicht}
\end{table}

\subsection{Video-Kategorisierung}

\begin{table}[H]
\centering
\begin{tabular}{@{}lcl@{}}
\toprule
Kategorie & Anzahl & Beschreibung \\
\midrule
Clean & 121 & Keine permanente zweite Person \\
Coach & 5 & Therapeut dauerhaft im Bild \\
\bottomrule
\end{tabular}
\caption{Video-Kategorisierung. Auch ``Clean''-Videos haben sporadische Multi-Person-Frames.}
\end{table}

\subsection{Modelle}

\begin{table}[H]
\centering
\begin{tabular}{@{}llll@{}}
\toprule
Modell & Entwickler & Keypoints & Selection-Strategie \\
\midrule
MediaPipe Pose & Google & 33 $\rightarrow$ 12 & Torso-Größe \\
MoveNet MultiPose & Google/TF & 17 $\rightarrow$ 12 & BBox-Fläche \\
YOLOv8-Pose Nano & Ultralytics & 17 $\rightarrow$ 12 & BBox-Fläche \\
\bottomrule
\end{tabular}
\caption{Evaluierte Modelle. Alle sind für Mobile-/On-Device-Nutzung optimiert.}
\end{table}

\subsection{Metrik: NMPJPE}

\textbf{Normalized Mean Per Joint Position Error} -- der durchschnittliche Gelenkfehler als Prozent der Torso-Länge (Schulter-Mitte zu Hüfte-Mitte).

\begin{itemize}[nosep]
    \item 10\% NMPJPE $\approx$ 5cm Fehler pro Gelenk (bei 50cm Torso)
    \item 15\% NMPJPE $\approx$ 7.5cm Fehler pro Gelenk
    \item $>$100\% NMPJPE = Modell trackt falsche Person
\end{itemize}

\subsection{Confidence-Thresholds}

\begin{table}[H]
\centering
\begin{tabular}{@{}lll@{}}
\toprule
Modell & Threshold & Begründung \\
\midrule
MediaPipe & 0.1 (Detection) & Default 0.5 zu strikt (29\% Failures) \\
MoveNet & 0.1 (Score) & Niedrig für Robustheit \\
YOLO & 0.3 (Joint Conf.) & Balance Robustheit/Qualität \\
\bottomrule
\end{tabular}
\caption{Verwendete Confidence-Thresholds für die Evaluation.}
\end{table}

%==============================================================================
\section{Modell-Ranking}
%==============================================================================

\subsection{Gesamtergebnis (Clean Data)}

\begin{table}[H]
\centering
\begin{tabular}{@{}lcccccc@{}}
\toprule
Modell & N Frames & Mean & \textbf{Median} & Std & P90 & P95 \\
\midrule
MediaPipe & 118.305 & 14.5\% & 11.2\% & 23.1\% & 17.0\% & 22.8\% \\
\rowcolor{bestcolor}
\textbf{MoveNet} & 118.472 & 14.8\% & \textbf{10.4\%} & 32.6\% & 16.8\% & 20.9\% \\
YOLO & 115.001 & 17.7\% & 11.3\% & 42.5\% & 20.2\% & 27.4\% \\
\bottomrule
\end{tabular}
\caption{Gesamtergebnis. Der \textbf{Median} ist aussagekräftiger als der Mean, da er robust gegen Ausreißer ist. MoveNet erreicht den besten Median (10.4\%).}
\end{table}

\textbf{Beobachtung:} YOLO's hoher Mean (17.7\%) und extreme Standardabweichung (42.5\%) deuten auf viele Ausreißer hin, nicht auf generell schlechtere Genauigkeit.

\subsection{Bereinigtes Ranking (ohne Person-Switch-Frames)}

Nach Entfernung von Frames mit $>$100\% Fehler (1.13\% aller Frames):

\begin{table}[H]
\centering
\begin{tabular}{@{}lcccc@{}}
\toprule
Modell & Raw Mean & \textbf{Bereinigt Mean} & Raw Std & \textbf{Bereinigt Std} \\
\midrule
\rowcolor{bestcolor}
\textbf{MoveNet} & 14.8\% & \textbf{11.5\%} & 32.6\% & \textbf{5.5\%} \\
MediaPipe & 14.5\% & 12.5\% & 23.1\% & 7.2\% \\
YOLO & 17.7\% & 12.9\% & 42.5\% & 6.8\% \\
\bottomrule
\end{tabular}
\caption{Nach Bereinigung sind alle Modelle ähnlich gut (11--13\%). Die Unterschiede kommen fast ausschließlich von Ausreißern.}
\end{table}

\subsection{Statistische Signifikanz}

\textbf{ANOVA:} $F = 316.30$, $p < 0.001$ -- signifikanter Unterschied zwischen Modellen.

\begin{table}[H]
\centering
\begin{tabular}{@{}lcccc@{}}
\toprule
Vergleich & Mean Diff & Cohen's $d$ & $p$-Wert & Signifikant \\
\midrule
MediaPipe vs MoveNet & $-0.25\%$ & $-0.009$ & 0.098 & Nein \\
MediaPipe vs YOLO & $-3.15\%$ & $-0.092$ & $<0.001$ & Ja \\
MoveNet vs YOLO & $-2.90\%$ & $-0.077$ & $<0.001$ & Ja \\
\bottomrule
\end{tabular}
\caption{Pairwise t-Tests (Bonferroni-korrigiert). MediaPipe und MoveNet sind statistisch nicht unterscheidbar. Beide sind signifikant besser als YOLO.}
\end{table}

%==============================================================================
\section{Kamera-Analyse (c17 vs c18)}
%==============================================================================

\subsection{Das Person-Switch-Problem}

\begin{table}[H]
\centering
\begin{tabular}{@{}lcccc@{}}
\toprule
Modell & c17 $>$100\% & c17 Rate & c18 $>$100\% & c18 Rate \\
\midrule
MediaPipe & 906 & 1.59\% & 187 & 0.31\% \\
MoveNet & 1.213 & 2.12\% & 95 & 0.16\% \\
\rowcolor{worstcolor}
YOLO & 1.500 & 2.79\% & 79 & 0.13\% \\
\bottomrule
\end{tabular}
\caption{Frames mit $>$100\% Fehler nach Kamera. c17 (frontal) hat \textbf{5--21x mehr} Person-Switch-Events als c18.}
\end{table}

\textbf{Erklärung:} c17 zeigt häufiger Situationen mit mehreren Personen im Bild (Therapeut, Hintergrund). Das BBox-basierte Selection bei MoveNet/YOLO versagt hier öfter.

\subsection{Outlier-Verteilung nach Schwellenwert}

\begin{table}[H]
\centering
\begin{tabular}{@{}lcccccc@{}}
\toprule
 & \multicolumn{3}{c}{\textbf{c17}} & \multicolumn{3}{c}{\textbf{c18}} \\
Modell & $>$30\% & $>$50\% & $>$100\% & $>$30\% & $>$50\% & $>$100\% \\
\midrule
MediaPipe & 1.974 & 1.239 & 906 & 1.759 & 855 & 187 \\
MoveNet & 1.614 & 1.309 & 1.213 & 1.126 & 375 & 95 \\
YOLO & 2.049 & 1.584 & 1.500 & 2.537 & 720 & 79 \\
\bottomrule
\end{tabular}
\caption{Anzahl Frames über verschiedenen NMPJPE-Schwellenwerten. c17 hat deutlich mehr extreme Ausreißer.}
\end{table}

\subsection{Bereinigter Kamera-Vergleich}

\begin{table}[H]
\centering
\begin{tabular}{@{}lcccc@{}}
\toprule
Modell & c17 Raw & c17 Bereinigt & c18 Raw & c18 Bereinigt \\
\midrule
MediaPipe & 15.5\% & 12.0\% & 13.6\% & 13.1\% \\
MoveNet & 16.8\% & \textbf{10.3\%} & 12.8\% & 12.6\% \\
YOLO & 21.3\% & 11.3\% & 14.5\% & 14.3\% \\
\bottomrule
\end{tabular}
\caption{Nach Bereinigung ist c17 (frontal) \textbf{1--2\% besser} als c18 (lateral) -- wie erwartet.}
\end{table}

%==============================================================================
\section{Rotations-Analyse}
%==============================================================================

\subsection{Datenverteilung und Kamera-Coverage}

Die Rotationswinkel sind \textbf{bimodal verteilt} -- Patienten stehen entweder frontal oder seitlich zur Kamera. Die beiden Kameras decken unterschiedliche Winkelbereiche ab:

\begin{table}[H]
\centering
\begin{tabular}{@{}lccc@{}}
\toprule
Rotation & c17 Frames & c18 Frames & Gesamt \\
\midrule
0--10° (frontal) & $\sim$15.700 & -- & $\sim$15.700 \\
10--20° & $\sim$13.700 & -- & $\sim$13.700 \\
20--30° & $\sim$1.900 & $\sim$15.100 & $\sim$17.000 \\
30--40° & $\sim$440 & $\sim$9.000 & $\sim$9.500 \\
40--50° & $\sim$1.100 & $\sim$1.100 & $\sim$2.200 \\
50--60° & $\sim$8.500 & $\sim$560 & $\sim$9.000 \\
60--70° & $\sim$14.800 & $\sim$2.700 & $\sim$17.500 \\
70--80° & -- & $\sim$16.400 & $\sim$16.400 \\
80--90° (lateral) & -- & $\sim$16.200 & $\sim$16.200 \\
\bottomrule
\end{tabular}
\caption{Rotations-Coverage nach Kamera. c17 deckt frontale (0--20°), c18 laterale (70--90°) Perspektiven ab. Der Überlappungsbereich (20--70°) hat weniger Daten.}
\end{table}

\textbf{Implikation:} Für eine vollständige 0--90° Rotationsanalyse müssen beide Kameras kombiniert werden. Einzelkamera-Analysen zeigen nur Teilbereiche.

%------------------------------------------------------------------------------
\subsection{Szenario 1: Kombinierte Rotation (c17 + c18, ohne >100\% Frames)}
%------------------------------------------------------------------------------

Durch Kombination beider Kameras erhalten wir den vollständigen Rotationsbereich:

\begin{table}[H]
\centering
\begin{tabular}{@{}lccccc@{}}
\toprule
Bucket & MediaPipe & MoveNet & YOLO & N Frames & Quelle \\
\midrule
\rowcolor{bestcolor}
0--10° & 10.1\% & \textbf{9.3\%} & 9.5\% & $\sim$15.300 & c17 only \\
\rowcolor{bestcolor}
10--20° & 10.3\% & \textbf{8.4\%} & 9.3\% & $\sim$13.300 & c17 only \\
20--30° & 10.5\% & 10.0\% & 10.8\% & $\sim$17.000 & c17+c18 \\
30--40° & 10.6\% & 9.9\% & 10.7\% & $\sim$9.500 & c17+c18 \\
40--50° & 11.0\% & 11.1\% & 13.9\% & $\sim$2.100 & c17+c18 \\
50--60° & 10.8\% & 10.2\% & 11.2\% & $\sim$9.000 & c17+c18 \\
60--70° & 12.5\% & 11.0\% & 12.3\% & $\sim$17.400 & c17+c18 \\
70--80° & 12.4\% & 12.1\% & 13.1\% & $\sim$16.300 & c18 only \\
\rowcolor{worstcolor}
80--90° & 13.3\% & 13.8\% & \textbf{14.9\%} & $\sim$16.200 & c18 only \\
\midrule
\textbf{Trend} & +32\% & +64\% & +60\% & & frontal$\rightarrow$lateral \\
\bottomrule
\end{tabular}
\caption{Kombinierte Rotationsanalyse (beide Kameras, bereinigt). Median NMPJPE steigt von frontal zu lateral um 32--64\%.}
\end{table}

\textbf{Beobachtungen:}
\begin{itemize}[nosep]
    \item Frontale Ansicht (0--20°) liefert beste Ergebnisse (Unterschied zwischen 0--10° und 10--20° ist $<$1\%)
    \item MoveNet zeigt stärksten relativen Anstieg (+64\%), aber startet mit niedrigstem Basiswert
    \item MediaPipe ist am stabilsten über alle Winkel (+32\% Anstieg)
    \item YOLO hat konsistent höhere Fehler bei lateraler Ansicht
\end{itemize}

%------------------------------------------------------------------------------
\subsection{Szenario 2: c17 Kamera (frontal-orientiert)}
%------------------------------------------------------------------------------

c17 zeigt primär frontale Perspektiven (0--70°), hat aber mehr Multi-Person-Probleme:

\begin{table}[H]
\centering
\begin{tabular}{@{}lcccc@{}}
\toprule
Bucket & MediaPipe & MoveNet & YOLO & N Frames \\
\midrule
\rowcolor{bestcolor}
0--10° & 10.1\% & \textbf{9.3\%} & 9.5\% & $\sim$15.300 \\
10--20° & 10.3\% & \textbf{8.4\%} & 9.3\% & $\sim$13.300 \\
20--30° & 11.3\% & 11.1\% & 11.1\% & $\sim$1.800 \\
30--40° & 11.4\% & 10.4\% & 12.9\% & $\sim$400 \\
40--50° & 10.8\% & 10.4\% & 13.4\% & $\sim$970 \\
50--60° & 10.7\% & 10.0\% & 10.9\% & $\sim$8.400 \\
60--70° & 12.4\% & 10.7\% & 11.9\% & $\sim$14.700 \\
\bottomrule
\end{tabular}
\caption{c17 Rotationsanalyse (ohne >100\% Frames). MoveNet erreicht besten Wert bei 10--20° (8.4\%), MediaPipe bei 0--10° (10.1\%).}
\end{table}

\textbf{c17-Besonderheit:} Die frontale Kamera hat 2.15\% Person-Switch-Frames vs nur 0.20\% bei c18. Nach Bereinigung zeigt c17 jedoch \textbf{bessere} Ergebnisse als c18 bei vergleichbaren Winkeln.

%------------------------------------------------------------------------------
\subsection{Szenario 3: c18 Kamera (lateral-orientiert)}
%------------------------------------------------------------------------------

c18 zeigt primär seitliche Perspektiven (20--90°), mit weniger Multi-Person-Störungen:

\begin{table}[H]
\centering
\begin{tabular}{@{}lcccc@{}}
\toprule
Bucket & MediaPipe & MoveNet & YOLO & N Frames \\
\midrule
20--30° & 10.5\% & 10.0\% & 10.8\% & $\sim$15.100 \\
30--40° & 10.6\% & \textbf{9.9\%} & 10.7\% & $\sim$9.000 \\
40--50° & 11.2\% & 11.7\% & 14.3\% & $\sim$1.100 \\
50--60° & 11.6\% & 11.9\% & 14.0\% & $\sim$550 \\
60--70° & 13.1\% & 12.7\% & 14.0\% & $\sim$2.700 \\
70--80° & 12.4\% & 12.1\% & 13.1\% & $\sim$16.300 \\
\rowcolor{worstcolor}
80--90° & 13.3\% & 13.8\% & \textbf{14.9\%} & $\sim$16.200 \\
\midrule
\textbf{Anstieg} & +27\% & +38\% & +39\% & & 20--30° $\rightarrow$ 80--90° \\
\bottomrule
\end{tabular}
\caption{c18 Rotationsanalyse (ohne >100\% Frames). Konsistenter Fehleranstieg zur seitlichen Ansicht.}
\end{table}

\textbf{Warum c18 für Rotationseffekt?} c18 hat weniger Confounding durch Person-Switch und zeigt den ``reinen'' Rotationseffekt klarer.

%------------------------------------------------------------------------------
\subsection{Vergleich: Raw vs Bereinigt}
%------------------------------------------------------------------------------

\begin{table}[H]
\centering
\begin{tabular}{@{}lccc|ccc@{}}
\toprule
 & \multicolumn{3}{c|}{\textbf{Raw (alle Frames)}} & \multicolumn{3}{c}{\textbf{Bereinigt (ohne >100\%)}} \\
Bucket & MP & MN & YOLO & MP & MN & YOLO \\
\midrule
0--10° (c17) & 10.1\% & 9.4\% & 9.6\% & 10.1\% & 9.3\% & 9.5\% \\
10--20° (c17) & 10.4\% & 8.5\% & 9.3\% & 10.3\% & 8.4\% & 9.3\% \\
80--90° (c18) & 13.3\% & 13.8\% & 14.9\% & 13.3\% & 13.8\% & 14.9\% \\
\bottomrule
\end{tabular}
\caption{Raw vs bereinigt zeigt minimale Unterschiede -- die Rotation-Buckets sind bereits relativ sauber.}
\end{table}

%------------------------------------------------------------------------------
\subsection{Zusammenfassung Rotationseffekt}
%------------------------------------------------------------------------------

\begin{table}[H]
\centering
\begin{tabular}{@{}lcccc@{}}
\toprule
Modell & Bestes Bucket & Schlechtestes & Absoluter Anstieg & Relativer Anstieg \\
\midrule
\rowcolor{bestcolor}
MediaPipe & 0--10° (10.1\%) & 80--90° (13.3\%) & +3.2\% & \textbf{+32\%} \\
MoveNet & 10--20° (8.4\%) & 80--90° (13.8\%) & +5.4\% & +64\% \\
YOLO & 10--20° (9.3\%) & 80--90° (14.9\%) & +5.6\% & +60\% \\
\bottomrule
\end{tabular}
\caption{MediaPipe zeigt die geringste Rotations-Sensitivität (+32\% relativ). Bestes Bucket variiert: MediaPipe bei 0--10°, MoveNet/YOLO bei 10--20° (Unterschied $<$1\%).}
\end{table}

%==============================================================================
\section{Selection-Robustheit}
%==============================================================================

\subsection{Coach-Impact-Analyse}

5 Videos enthielten einen Therapeuten dauerhaft im Bild. Dies offenbart kritische Unterschiede:

\begin{table}[H]
\centering
\begin{tabular}{@{}lcccc@{}}
\toprule
Modell & Clean Mean & Coach Mean & Anstieg & Selection \\
\midrule
\rowcolor{bestcolor}
\textbf{MediaPipe} & 14.5\% & 44.8\% & \textbf{+209\%} & Torso-Größe \\
MoveNet & 14.8\% & 64.9\% & +340\% & BBox-Fläche \\
YOLO & 17.7\% & 66.1\% & +274\% & BBox-Fläche \\
\bottomrule
\end{tabular}
\caption{MediaPipe's Torso-basierte Selection ist $\sim$2x robuster als BBox-Selection.}
\end{table}

\subsection{Warum Torso-Selection besser funktioniert}

\begin{itemize}[nosep]
    \item \textbf{Torso-Größe} (Schulter-Hüfte-Abstand) korreliert mit Kamera-Distanz
    \item \textbf{BBox-Fläche} misst ``Ausbreitung'' (Armposition), nicht tatsächliche Größe
    \item Ein Therapeut mit ausgestreckten Armen hat große BBox aber kleinen Torso
\end{itemize}

%==============================================================================
\section{Per-Joint-Analyse}
%==============================================================================

\begin{table}[H]
\centering
\begin{tabular}{@{}lccc@{}}
\toprule
Körperregion & MediaPipe & MoveNet & YOLO \\
\midrule
Schultern & 7--8\% & 8--10\% & 8--10\% \\
Ellbogen & 9\% & 10\% & 10\% \\
Handgelenke & 9--10\% & 11--12\% & 12--13\% \\
\rowcolor{worstcolor}
Hüften & \textbf{16--17\%} & 10--11\% & 14--15\% \\
Knie & 7--8\% & 6\% & 7--8\% \\
Knöchel & 12\% & 10\% & 10\% \\
\bottomrule
\end{tabular}
\caption{Median NMPJPE nach Körperregion. MediaPipe zeigt auffällig hohe Hüft-Fehler.}
\end{table}

\textbf{Beobachtung:} MediaPipe hat überdurchschnittlich hohe Hüft-Fehler (16--17\% vs 10--15\% bei anderen). Dies ist ein architekturspezifisches Muster.

%==============================================================================
\section{Übungs-Analyse}
%==============================================================================

\begin{table}[H]
\centering
\begin{tabular}{@{}lcccc@{}}
\toprule
Übung & MediaPipe & MoveNet & YOLO & Schwierigkeit \\
\midrule
Ex1 & 11.4\% & 10.2\% & 10.4\% & Leicht \\
\rowcolor{bestcolor}
Ex2 & 10.8\% & \textbf{9.4\%} & 10.4\% & Leichteste \\
\rowcolor{worstcolor}
Ex3 & 12.6\% & 13.6\% & \textbf{14.2\%} & Schwierigste \\
Ex4 & 11.3\% & 11.2\% & 12.0\% & Mittel \\
Ex5 & 11.0\% & 11.2\% & 13.1\% & Mittel \\
Ex6 & 11.3\% & 10.2\% & 11.6\% & Leicht \\
\bottomrule
\end{tabular}
\caption{Median NMPJPE nach Übung. Ex3 ist für alle Modelle am schwierigsten.}
\end{table}

%==============================================================================
\section{Zusammenfassung und Empfehlungen}
%==============================================================================

\subsection{Zentrale Erkenntnisse}

\begin{enumerate}[nosep]
    \item \textbf{Modell-Ranking:} MoveNet $\approx$ MediaPipe $>$ YOLO (bereinigt: 11.5\% vs 12.5\% vs 12.9\%)
    \item \textbf{Ausreißer dominieren:} Die Mean-Unterschiede kommen fast ausschließlich von Person-Switch-Events
    \item \textbf{c17-Problem:} Die frontale Kamera hat 10x mehr Multi-Person-Probleme
    \item \textbf{Selection-Strategie kritisch:} Torso-Selection (MediaPipe) ist 2x robuster als BBox-Selection
    \item \textbf{Rotationseffekt messbar:} +27--39\% Fehler bei seitlicher vs schräger Ansicht
\end{enumerate}

\subsection{Empfehlungen für Anwendungen}

\begin{table}[H]
\centering
\begin{tabular}{@{}lll@{}}
\toprule
Szenario & Empfehlung & Begründung \\
\midrule
Single-Person garantiert & MoveNet & Beste Genauigkeit (10.4\% Median) \\
Multi-Person möglich & MediaPipe & 2x robustere Selection \\
Ressourcen-limitiert & YOLO Nano & Schnellste Inferenz \\
\bottomrule
\end{tabular}
\end{table}

\subsection{Implementierungshinweise}

\begin{itemize}[nosep]
    \item \textbf{Multi-Person-Warnung:} App sollte warnen wenn $>$1 Person erkannt
    \item \textbf{Seitliche Ansicht:} Fehler steigt um $\sim$30\%, User zur Neupositionierung auffordern
    \item \textbf{Extreme-Frame-Detection:} NMPJPE $>$100\% automatisch als ungültig markieren
\end{itemize}

%==============================================================================
\section{Limitationen}
%==============================================================================

\begin{itemize}[nosep]
    \item \textbf{Bimodale Rotation:} Keine kontinuierliche Rotation-Analyse möglich
    \item \textbf{Kleine Coach-Stichprobe:} Selection-Robustheit basiert auf nur 5 Videos
    \item \textbf{Dataset-spezifisch:} Ergebnisse gelten für Rehabilitations-Kontext
    \item \textbf{Nur mobile Varianten:} Keine Aussage über Server-Modelle
\end{itemize}

%==============================================================================
\section*{Anhang: Datengrundlage}
%==============================================================================

\begin{table}[H]
\centering
\small
\begin{tabular}{@{}ll@{}}
\toprule
Datei & Beschreibung \\
\midrule
\texttt{frame\_level\_data.csv} & 363.529 Frames mit allen Metriken \\
\texttt{rotation\_analysis.csv} & NMPJPE pro 10°-Bucket \\
\texttt{outlier\_analysis.csv} & Video-Level Ausreißer \\
\texttt{per\_joint\_analysis.csv} & Joint-Level Fehler \\
\texttt{summary\_statistics.json} & Maschinenlesbare Zusammenfassung \\
\bottomrule
\end{tabular}
\end{table}

\vspace{1em}
\textbf{Reproduktion:}
\begin{verbatim}
cd C:/Users/Eren/bachelor
.venv/Scripts/python analysis/comprehensive_analysis.py
.venv/Scripts/python analysis/extended_analysis.py
\end{verbatim}

\end{document}
