\documentclass[11pt,a4paper]{article}
\usepackage[utf8]{inputenc}
\usepackage[margin=2cm]{geometry}
\usepackage{booktabs}
\usepackage{graphicx}
\usepackage{xcolor}
\usepackage{hyperref}
\usepackage{float}
\usepackage{amsmath}
\usepackage{enumitem}

\definecolor{bestcolor}{RGB}{200, 230, 200}
\definecolor{highlight}{RGB}{255, 255, 200}

\setlength{\parindent}{0pt}
\setlength{\parskip}{0.5em}

\title{\textbf{Pose Estimation Evaluation}\\[0.3em]\large Übersicht für Betreuer-Gespräch}
\author{[Name]}
\date{Januar 2026}

\begin{document}
\maketitle

%==============================================================================
\section*{Executive Summary}
%==============================================================================

\begin{table}[H]
\centering
\begin{tabular}{@{}ll@{}}
\toprule
\textbf{Aspekt} & \textbf{Ergebnis} \\
\midrule
Datenbasis & 363.529 Frames, 126 Videos, MoCap Ground Truth \\
Modelle & MediaPipe, MoveNet, YOLOv8-Pose (alle mobile-optimiert) \\
\midrule
Accuracy & MediaPipe $\approx$ MoveNet (\textbf{nicht signifikant}, $p=0.098$) \\
Rotation Robustheit & MediaPipe am besten (+31\% vs +54\% Degradation) \\
Multi-Person Robustheit & MediaPipe 2x besser (Torso- vs BBox-Selection) \\
Temporal Stability & MoveNet am stabilsten (42\% weniger Jitter) \\
\midrule
\rowcolor{highlight}
\textbf{Haupterkenntnis} & \textbf{Kein Modell dominiert -- Trade-offs je nach Szenario} \\
\bottomrule
\end{tabular}
\end{table}

%==============================================================================
\section{Methodik}
%==============================================================================

\textbf{Dataset:} REHAB24-6 (öffentlich, Zenodo)
\begin{itemize}[nosep]
    \item 21 Rehabilitations-Patienten, 6 Physiotherapie-Übungen
    \item 2 Kameras: c17 (frontal, 0--70°), c18 (lateral, 20--90°)
    \item Ground Truth: Optisches Motion Capture System
\end{itemize}

\textbf{Metrik:} NMPJPE (Normalized Mean Per Joint Position Error)
\begin{itemize}[nosep]
    \item Fehler normalisiert auf Torso-Länge (vergleichbar über Personen)
    \item 10\% NMPJPE $\approx$ 5cm Fehler pro Gelenk (bei 50cm Torso)
\end{itemize}

%==============================================================================
\section{Ergebnisse im Detail}
%==============================================================================

\subsection{Accuracy (bereinigt, ohne Ausreißer)}

\begin{table}[H]
\centering
\begin{tabular}{@{}lccc@{}}
\toprule
Modell & Mean NMPJPE & Median NMPJPE & Std \\
\midrule
MoveNet & 11.5\% & 10.4\% & 5.5\% \\
MediaPipe & 12.5\% & 11.2\% & 7.2\% \\
YOLO & 12.9\% & 11.3\% & 6.8\% \\
\midrule
\multicolumn{4}{l}{\small MediaPipe vs MoveNet: $p=0.098$ (nicht signifikant), Cohen's $d=0.009$} \\
\bottomrule
\end{tabular}
\end{table}

\subsection{Rotation Robustheit}

\begin{table}[H]
\centering
\begin{tabular}{@{}lcccc@{}}
\toprule
Modell & Frontal (0--20°) & Lateral (80--90°) & Relativer Anstieg \\
\midrule
\rowcolor{bestcolor}
\textbf{MediaPipe} & 10.2\% & 13.3\% & \textbf{+31\%} \\
MoveNet & 9.0\% & 13.8\% & +54\% \\
YOLO & 9.4\% & 14.9\% & +58\% \\
\bottomrule
\end{tabular}
\caption{MediaPipe ist am robustesten gegenüber Rotation.}
\end{table}

\subsection{Multi-Person Robustheit (Selection-Strategie)}

\begin{table}[H]
\centering
\begin{tabular}{@{}lcccc@{}}
\toprule
Modell & Selection & Clean Mean & Coach Mean & Anstieg \\
\midrule
\rowcolor{bestcolor}
\textbf{MediaPipe} & Torso-Größe & 14.5\% & 44.8\% & \textbf{+209\%} \\
MoveNet & BBox-Fläche & 14.8\% & 64.9\% & +340\% \\
YOLO & BBox-Fläche & 17.7\% & 66.1\% & +274\% \\
\bottomrule
\end{tabular}
\caption{MediaPipe's Torso-basierte Selection ist $\sim$2x robuster. (N=5 Coach-Videos)}
\end{table}

\subsection{Temporal Stability (Jitter)}

\begin{table}[H]
\centering
\begin{tabular}{@{}lccc@{}}
\toprule
Modell & Mean Jitter & Median Jitter & Interpretation \\
\midrule
\rowcolor{bestcolor}
\textbf{MoveNet} & \textbf{1.06\%} & 0.42\% & Stabilste Predictions \\
YOLO & 1.12\% & 0.38\% & Ähnlich stabil \\
MediaPipe & 1.51\% & 0.53\% & 42\% mehr Variation \\
\bottomrule
\end{tabular}
\caption{MoveNet liefert die stabilsten Frame-zu-Frame Predictions.}
\end{table}

%==============================================================================
\section{Vergleich mit Literatur}
%==============================================================================

\begin{table}[H]
\centering
\small
\begin{tabular}{@{}p{3.5cm}p{4cm}p{4cm}@{}}
\toprule
\textbf{Paper} & \textbf{Was sie machen} & \textbf{Was wir ergänzen} \\
\midrule
UCO Dataset (2023) & 8 Modelle, gesunde Probanden & MoveNet, echte Patienten \\
Baldinger (2025) & OpenPose, 4 Winkel & 3 mobile Modelle, 0--90° \\
Ullah (2025) & MediaPipe für Scoring & Modellvergleich, Jitter \\
\midrule
\textbf{Gap} & Selection-Strategien nicht verglichen & \textbf{Torso vs BBox quantifiziert} \\
\textbf{Gap} & Temporal Stability nicht untersucht & \textbf{Jitter-Analyse} \\
\bottomrule
\end{tabular}
\end{table}

%==============================================================================
\section{Geplante Thesis-Richtung}
%==============================================================================

\textbf{Arbeitstitel:} ``Evaluating Mobile Pose Estimation Models for Home-Based Rehabilitation: Accuracy, Stability, and Robustness''

\textbf{Research Questions:}
\begin{enumerate}[nosep]
    \item Wie genau sind mobile HPE-Modelle auf echten Reha-Daten?
    \item Wie stabil sind die Predictions über Zeit?
    \item Wie robust sind die Modelle bei suboptimalen Bedingungen?
    \item Welche praktischen Empfehlungen ergeben sich?
\end{enumerate}

\textbf{Contributions:}
\begin{itemize}[nosep]
    \item Erste MoveNet-Evaluation auf Reha-Daten mit MoCap Ground Truth
    \item Erste systematische Temporal Stability Analyse für mobile HPE
    \item Quantifizierung von Selection-Strategien bei Multi-Person
    \item Praktische Guidelines für mobile Physiotherapie-Apps
\end{itemize}

%==============================================================================
\section{Empfehlung für Previa Health}
%==============================================================================

\begin{table}[H]
\centering
\begin{tabular}{@{}lll@{}}
\toprule
\textbf{Szenario} & \textbf{Empfehlung} & \textbf{Begründung} \\
\midrule
\rowcolor{bestcolor}
Home-Based Reha & \textbf{MediaPipe} & Robuster bei Rotation \& Multi-Person \\
Kontrollierte Umgebung & MoveNet & Etwas genauer, stabiler \\
\bottomrule
\end{tabular}
\end{table}

%==============================================================================
\section{Limitationen}
%==============================================================================

\begin{itemize}[nosep]
    \item Bimodale Rotationsverteilung (keine kontinuierliche Rotation während Übungen)
    \item Nur 5 Coach-Videos für Multi-Person-Analyse
    \item Nur mobile Modellvarianten getestet
\end{itemize}

%==============================================================================
\section*{Offene Frage an Betreuer}
%==============================================================================

\fbox{\parbox{0.95\textwidth}{
Ist der breite Evaluations-Fokus (Accuracy + Stability + Robustness) tragfähig für eine Bachelorarbeit, oder sollte stärker auf einen einzelnen Aspekt fokussiert werden?
}}

\end{document}
